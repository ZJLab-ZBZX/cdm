
\documentclass[12pt]{article}
\usepackage[landscape]{geometry}
\usepackage{geometry}
\geometry{a5paper,scale=0.98}
\pagestyle{empty}
\usepackage{booktabs}
\usepackage{amsmath}
\usepackage{upgreek}
\usepackage{amssymb}
\usepackage{xcolor}
\begin{document}
\makeatletter
\renewcommand*{\@textcolor}[3]{%
  \protect\leavevmode
  \begingroup
    \color#1{#2}#3%
  \endgroup
}
\makeatother
\begin{displaymath}
\begin{align*}k=J-m-1\qquad & \mathcolor[RGB]{0,0,15}{\texttt{sumP}} \mathcolor[RGB]{0,0,30}{=} \mathcolor[RGB]{0,0,45}{\texttt{c[J-m-1]}k=J-m-2\qquad} & \mathcolor[RGB]{0,0,60}{\texttt{sumP}} \mathcolor[RGB]{0,0,75}{=} \mathcolor[RGB]{0,0,90}{\texttt{c[J-m-1]p}(1+\eta_2)+c[J-m-2](1+\eta_1)k=J-m-3\qquad} & \mathcolor[RGB]{0,0,105}{\texttt{sumP}} \mathcolor[RGB]{0,0,120}{=} \mathcolor[RGB]{0,0,135}{\texttt{c[J-m-1]p}^2(1+\eta_4)+\texttt{c[J-m-2]p}(1+\eta_3)+&+\texttt{c[J-m-3]}(1+\eta_1)k=0\qquad} & \mathcolor[RGB]{0,0,150}{\texttt{sumP}} \mathcolor[RGB]{0,0,165}{=} \mathcolor[RGB]{0,0,180}{\texttt{c[J-m-1]p}^{J-m-1}(1+\eta_{2J-2m-2})+&+\sum_{k=0}^{J-m-2}\texttt{c[k]p}^k(1+\eta_{2k+1}).\begin{align*}}
\end{displaymath}
\end{document}

